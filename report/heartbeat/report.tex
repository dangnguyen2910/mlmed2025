\documentclass[conference]{IEEEtran}
\usepackage{cite}

\title{ECG Heartbeat Classification for Arrhythmia Detection: A Machine Learning Approach}
\author{
	\IEEEauthorblockN{Nguyen Hai Dang - 22BI13073}
	\IEEEauthorblockA{
		\textit{Department of Information and Communication Technology} \\ 
		\textit{University of Science and Technology of Hanoi} \\ 
	}
}


\begin{document}
	\maketitle
	\begin{abstract}	
		
	\end{abstract}
	
	\begin{IEEEkeywords}
		ECG, arrhythmia, machine learning, random forest
	\end{IEEEkeywords}
	
	\section{Introduction}
	
	Arrhythmia is a term refers to any problem that relate to abnormal heartbeat rhythm. There are mainly 2 types of arrhythmia. A heart beats too fast (more than 100 beats per minute when resting) is called tachycardia \cite{tachycardia}. A heart beats too slow (less than 60 beats per minutes when resting) is called bradycardia \cite{bradycardia}. Tachycardia can cause fainting and thrombosis (i.e., blood clots blocking blood vessels). Bradycardia, although it can be normal for most cases, especially for healthy people or athletes, it could lead to many symptoms such as chest pain, confusion, memory problems, etc. Therefore, being able to classify heartbeat from ECG is crucial. 
	
	 
	\section{Dataset}
	\subsection{Preliminary Information}
	\subsection{Data Analysis}
	\section{Experiment}
	\section{Conclusion}


	\begin{thebibliography}{9}
	\bibitem{tachycardia} Awtry EH, Jeon C, Ware MG (2006). "Tachyarrhythmias". Blueprints Cardiology (2nd ed.). Malden, Mass.: Blackwell. p. 93. ISBN 9781405104647.
	
	\bibitem{bradycardia} Hafeez Y, Grossman SA (9 August 2021). "Sinus bradycardia". StatPearls [Internet]. Treasure Island (FL): StatPearls Publishing. PMID 29630253. Retrieved 16 January 2022.
		
	\bibitem{kachuee} M. Kachuee, S. Fazeli, and M. Sarrafzadeh, “ECG Heartbeat Classification: A Deep Transferable Representation,” in 2018 IEEE International Conference on Healthcare Informatics (ICHI), Jun. 2018, pp. 443–444. doi: 10.1109/ICHI.2018.00092.
	
	\bibitem{guan} J. Guan, W. Wang, P. Feng, X. Wang and W. Wang, "Low-Dimensional Denoising Embedding Transformer for ECG Classification," ICASSP 2021 - 2021 IEEE International Conference on Acoustics, Speech and Signal Processing (ICASSP), Toronto, ON, Canada, 2021, pp. 1285-1289, doi: 10.1109/ICASSP39728.2021.9413766. 
	
	

	\end{thebibliography}
	
  
  
\end{document}